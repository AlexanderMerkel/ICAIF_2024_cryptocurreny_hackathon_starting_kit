
\documentclass[11pt]{article}

\usepackage{amsmath}
\usepackage{amssymb}
\usepackage{amsthm}
\usepackage[english]{babel}
\usepackage{color}
\usepackage{dsfont}
\usepackage{environ}
\usepackage[utf8x]{inputenc}
\usepackage{mathrsfs}
\usepackage{libertine}
\usepackage[T1]{fontenc}
\usepackage{setspace}
\usepackage{mathtools}
\usepackage{stmaryrd}
\usepackage{natbib}
\usepackage{url}
\usepackage[a4paper, total={7in, 10in}]{geometry}


\expandafter\def\expandafter\normalsize\expandafter{%
\normalsize
\setlength\abovedisplayskip{10pt}
\setlength\belowdisplayskip{10pt}
\setlength\abovedisplayshortskip{10pt}
\setlength\belowdisplayshortskip{13pt}
}

\newcommand{\C}{\mathbb{C}}
\newcommand{\E}{\mathbb{E}}
\newcommand{\F}{\mathbb{F}}
\newcommand{\N}{\mathbb{N}}
\renewcommand{\P}{\mathbb{P}}
\newcommand{\Q}{\mathbb{Q}}
\newcommand{\R}{\mathbb{R}}
\renewcommand{\S}{\mathbb{S}}
\newcommand{\T}{\mathbb{T}}
\newcommand{\V}{\mathbb{V}}
\newcommand{\Z}{\mathbb{Z}}

\newcommand{\cA}{\mathcal{A}}
\newcommand{\cB}{\mathcal{B}}
\newcommand{\cC}{\mathcal{C}}
\newcommand{\cD}{\mathcal{D}}
\newcommand{\cE}{\mathcal{E}}
\newcommand{\cF}{\mathcal{F}}
\newcommand{\cH}{\mathcal{H}}
\newcommand{\cI}{\mathcal{I}}
\newcommand{\cJ}{\mathcal{J}}
\newcommand{\cL}{\mathcal{L}}
\newcommand{\cM}{\mathcal{M}}
\newcommand{\cN}{\mathcal{N}}
\newcommand{\cQ}{\mathcal{Q}}
\newcommand{\cS}{\mathcal{S}}
\newcommand{\cT}{\mathcal{T}}
\newcommand{\cU}{\mathcal{U}}
\newcommand{\cV}{\mathcal{V}}
\newcommand{\cW}{\mathcal{W}}
\newcommand{\cY}{\mathcal{Y}}
\newcommand{\cZ}{\mathcal{Z}}

\newcommand{\fF}{\mathfrak{F}}
\newcommand{\fG}{\mathfrak{G}}
\newcommand{\fh}{\hat{h}}
\newcommand{\fM}{\mathfrak{M}}
\newcommand{\fN}{\mathfrak{N}}
\newcommand{\fO}{\mathfrak{O}}
\newcommand{\fS}{\mathfrak{S}}
\newcommand{\fw}{\mathfrak{w}}

\newcommand{\rD}{\mathrm{D}}
\newcommand{\rd}{\mathrm{d}}
\newcommand{\rF}{\mathrm{F}}
\newcommand{\rI}{\mathrm{I}}
\newcommand{\rL}{\mathrm{L}}

\newcommand{\sDM}{\mathscr{DM}}
\newcommand{\DM}{{decision maker}}
\newcommand{\intS}{\mathring{\S}}

\newcommand{\argdot}{\,\cdot\,}
\newcommand{\equi}{\gamma_{\mathrm{eq}}}
\newcommand{\hEqui}{h_{\mathrm{eq}}}
\newcommand{\vEqui}{v_{\mathrm{eq}}}
\newcommand{\pEqui}{p_{\mathrm{eq}}}

\newcommand{\comp}{\mathsf{c}}
\newcommand{\dom}{\mathrm{dom}\,}
\newcommand{\eps}{\varepsilon}
\newcommand{\gph}{\mathrm{gph}\,}
\newcommand{\ind}{\mathbb{I}}
\newcommand{\Leb}{\mathrm{Leb}}

\DeclareMathOperator{\sign}{sign}
\DeclareMathOperator{\metric}{d}
\DeclareMathOperator{\diag}{diag}
\DeclareMathOperator{\trace}{tr}
\DeclareMathOperator{\Diff}{D}
\DeclareMathOperator*{\esssup}{ess\,sup}
\DeclareMathOperator*{\essinf}{ess\,inf}

\theoremstyle{plain}
\newtheorem{theorem}{Theorem}[section]
\newtheorem{proposition}[theorem]{Proposition}
\newtheorem{corollary}[theorem]{Corollary}
\newtheorem{lemma}[theorem]{Lemma}
\newtheorem{definition}[theorem]{Definition}
\newtheorem{assumption}{Assumption}
\newtheorem{notation}{Notation}
\newtheorem{claim}{Claim}

\theoremstyle{definition}
\newtheorem{remark}[theorem]{Remark}
\newtheorem{example}[theorem]{Example}

\newcommand{\close}{\hspace*{\fill}$\diamond$}
\newcommand{\closeEqn}{\tag*{$\diamond$}}


\title{Approaches}
\author{Alexander Merkel}

\begin{document}
\maketitle

\section{The OU-OU model}
We model the three-dimensional $\log$-returns $R = (R^{1},R^{2},R^{3})$ as
\begin{equation}
    \rd R_{t} = \bigl[\mu - \frac{1}{2} \diag\bigl(\Sigma_{t}\Sigma_{t}^{\top}\bigr)\bigr] \rd t 
    + \Sigma_{t} \rd B_{t},\quad R_{0} = r_{0},
\end{equation}
where $\mu\in \R^{3}$ is the drift, $\sigma_{t} \in \R^{3}$ is the volatility, and $B = (B^{1},B^{2},B^{3})$ is a standard Brownian motion in $\R^{3}$. We assume that the volatility is given by
and most importantly with $X :=\log(\Sigma)$ the log-volatility process, we have
\begin{align*}
    \rd Y_{t} &= - \beta Y_{t} \rd t + \Sigma_{Y}\rd W^{1}_{t}, \quad X_{0} = x_{0},\\
    \rd X_{t} &= \lambda\bigl( Y_{t} - X_{t}\bigr) \rd t + \Sigma_{X}\rd W^{2}_{t},
    \quad X_{0} = x_{0},
\end{align*}
and $X, Y \in \R^{3 \times 3}$ are $3\times 3$-matrix-valued processes, and $W^{1}, W^{2}$
are independent standard Brownian motions in $\R^{3}$.

The parameters $\beta, \lambda > 0$ are the mean-reversion rates,
and $\Sigma_{Y}, \Sigma_{X} \in \cS_{++}^{3}$ are the covariance matrices of the OU-OU model.
The Brownian motions $W^{1}, W^{2}$ are independent standard Brownian motions in $\R^{3}$.

\begin{example}
    In the case of a single asset, we have $R = (R^{1})$ and $\Sigma = \sigma > 0$.
    The parameters of the OU-OU model in the reference are e.g.
    \begin{align*}
        \Sigma_{X} = 20,\quad
        \Sigma_{Y} = 0.625,\quad
        \lambda = 210,\quad
        \beta = 2.5.
    \end{align*}
\end{example}

\subsubsection*{Fitting the model}
We can obtain the $\log$-volatility process $X$ by using
\begin{equation}
    \langle R_{t} \rangle_{t} 
    = \int_{0}^{t} \Sigma_{s}\Sigma_{s}^{\top} \rd s,
\end{equation}
together with the approximation
\begin{equation}
    \langle R_{t} \rangle_{t} \approx \sum_{n=0}^{N-1} \bigl(R_{t_{n+1}} - R_{t_{n}}\bigr)\bigl(R_{t_{n+1}} - R_{t_{n}}\bigr)^{\top},
\end{equation}
for the covariation of the $\log$-returns $R$.
Hence we can estimate the $\log$-volatility process $X$ by



\subsection{As discrete-time model}
Discretizing the OU-OU model, on an equidistant grid $t_{n} = n\Delta t$ with $n\in \N$ and $\Delta t > 0$, we obtain
\begin{align}
    R_{t_{n+1}} &= R_{t_{n}} + \bigl[\mu - \frac{1}{2}\diag\bigl(\Sigma_{t_{n}}\Sigma_{t_{n}}^{\top}\bigr)\bigr] + \Sigma_{t_{n}}Z_{n},\\
    Y_{t_{n+1}} &= (1-\beta Y_{t_{n}})  + \Sigma_{Y}Z^{1}_{n},\\
    X_{t_{n+1}} &= X_{t_{n}} + \lambda\bigl(Y_{t_{n}} - X_{t_{n}}\bigr) + \Sigma_{X}Z^{2}_{n},
\end{align}
where in the application the parameters correspond to the 1-hourly estimation of the model.

\section{A GARCH model}
In $GARCH(1,1)$, the volatility process is given by
\begin{align*}
    r_{t} &= \mu + \sigma_{t}Z_{t},\\
    \Sigma_{t}^{2} &= \omega + \alpha r_{t-1} r_{t-1}^{T} + \beta \Sigma_{t-1}\Sigma_{t-1}^{T},
\end{align*}
where we test $Z\sim \cN(0,1)$ or $Z\sim t_{1}$.



\bibliographystyle{plain}
\bibliography{References}

\end{document}
